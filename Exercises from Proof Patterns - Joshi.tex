
\documentclass{article} 
\usepackage{amsmath} 
\usepackage{amssymb}
\usepackage{xcolor}
\usepackage[left=1in, right=1in]{geometry}
\renewcommand\thesubsection{Exercise \arabic{section}.\arabic{subsection}}
\newcommand{\exercise}[1]{\subsection{\normalfont #1}}
\newcommand{\solution}{\indent\indent \textbf{Solution: }}

%\title{Brief Article} \author{The Author}
%\maketitle

\begin{document}

\setcounter{section}{1}
\section{Double Counting}

\exercise{Let $a$ and $b$ be integers. Develop a formula for
$$\sum^n_{j=1} a + jb.$$ }
\solution Using the double counting pattern, the sum can also be represented as 
$$S_n =\sum^n_{j=1}a + (N+1)b - jb$$
Then we have 
$$2S_n = \sum^n_{j=1} 2a + (N+1)b$$
$$S = Na + \tfrac{N(N+1)}{2}b$$

\exercise{How many vertices does an icosahedron have?}
\solution{The number of vertex-faces is 60 since there are 20 faces and 3 vertices per face. Then 
$$5V = 60$$
since each vertex is connected to 5 faces. The number of vertices is 12.}

\exercise{If $p$ is a prime and not 2, and $a$ is an integer, show that 2 divides into $a^p - a$. (Try to construct a bijection on the set of size $p$ orbits which pairs them.)}
\solution{We can regard $(a^p -a)/p$ as the number of size $p$ orbits on strings of length $p$ chosen from $a$ different characters. Since $p \mid a^p -a$ and 2 doesn't divide $p$, 2 divides $a^p-a$ if and only if 2 divides $(a^p-a)/p$. \\ \indent Let $A$ be a size $a$ set of characters. Define a bijection $k:A\to A$ such that at most one element of $A$ is mapped to itself and $k(k(x)) = x$ $\forall x$. If $S$ is a string of length $p$, denote $S_j$ as the $j^{th}$ character of the string. Let $K$ be a function on the set of strings of length $p$ such that if $T = K(S)$ then $T_j = k(S_j)$ for $j \in [1,p]$. $K$ is a bijection because $k$ is a bijection. Additionally $K$ has the property that for any $S$ in an orbit of length greater than 1, $K(S)$ is in a different orbit, and if strings $S$ and $T$ are in the same orbit, then $K(S)$ and $K(T)$ are in the same orbit. Therefore $K$ also defines a bijection on the set of orbits. $K$ is a pairing on the set of orbits of length $p$, so the set of orbits of size $p$, of which there are $(a^p - a)/p$, is divisible by two. Then $2\mid a^p -a$.}

\exercise{Suppose we have three sorts of jewels and apply the arguments for Vandermonde's identity, what formula do we find?}
\solution
There are two ways to count the ways to choose $r$ jewels from $l$ jewels of type 1, $m$ jewels of type 2, and $n$ jewels of type 3. The first is to take ${l+m+n \choose r}$. Or, by counting sequentially
\[\sum^{r}_{j=0} \sum^{r-j}_{k=0} {l\choose j} {m\choose k} {n\choose r-j-k}. \]
Equating the two experessions gives the formula
\[ {l+m+n\choose r} = \sum^{r}_{j=1} \sum^{r-j}_{k=1} {l\choose j} {m\choose k} {n\choose r-j-k}. \]

\newpage
\section{The Pigeonhole Principle}

\exercise{Show that if a lossless compression algorithm exists for any file, then it is possible to represent any file by a single bit of information.}
\solution
A \textit{compression algorithm} maps binary sequences of length $m$ to binary sequences of length $n$, where $m>n$. A \textit{lossless compression algorithm} is a compression algorithm which is an injective map. Any file can be represented by a single bit of information if there is an injective function from the set of binary sequences to the set of binary sequences of length 1.

Let $x\in \{0, 1\}^k$ be an arbitrary binary sequence and suppose $T:\{0,1\}^\omega \to \{0, 1\}^\omega$ is a lossless compression algorithm. The length of $T(x)$ must be less than the length of $x$, and $T$ is injective, so $T^k$ is a composition of injective functions and $T^k(x)$ is at most 1 bit in length. 

\exercise{Show that there are two residents of London with the same number of hairs on their head.}
\solution
Let $L$ be the set of residents of London. The cardinality of $L$ is 9 million. Let $h:L\to \mathbb{Z}$ be the map from each resident to the number of hairs on his or her head. Assuming that each resident has less than 9 million hair follicles (the average is 150,000) then $\vert Im(h)\vert < \vert L \vert$. Therefore the map $h': L\to Im(h)$ is a map from a set of higher cardinality to lower cardinality. By the pigeonhole principle $h$ is not injective, and there must be residents $a, b \in L, a\neq b$ such that $a$ and $b$ have the same number of hair follicles, $h(a)=h(b)$.

\exercise{Suppose $N$ people are at a party. Some have met before and some have not. Show that there are two people who have met the same number of other people before.} 
\solution
Let $S$ be the set of people at the party. The map $m:S\to \mathbb{Z}$ represents the number of other people at the party that each person has met before. Clearly $m(x) \in [0, N-1]$ since each person has met at least no other person, and there are $N-1$ other people at the party. Suppose $m(x) = 0$ for some $x\in S$. Then for all $y\in S, m(y) \neq N-1$. The cardinality of the image of $m$ is less than $N$, and by the pigeonhole principle the map $m$ is not injective. 

If $m(x) \neq 0$ for all $x\in S$, then $Im(m)$ is a subset of integers from 1 to $N-1$ and has cardinality less than $N$. The pigeonhole principle says that the map is not injective. 

\exercise{Suppose we take a set of 101 different integers between 1 and 200. Show that there is a pair such that one divides the other.}
\solution 
Let a \textit{dividing pair} be a pair of integers such that one divides the other. Every integer is either a power of 2 or an odd number times a power of 2, so every integer between 1 to 200 is of the form $(2n+1)2^m$, where $n$ and $m$ are natural numbers including zero. If two different integers have the same odd coefficient, they take on the form $k2^m$ and $k2^{m'}$. If $m' < m$ then $k2^m = k2^{m'}(2^{m-m'})$ in which case $k2^{m'}$ divides $k2^m$. Otherwise if $m' > m$ then $k2^{m'} = k2^m(k2^{m'-m})$ and $k2^m$ divides $k2^{m'}$. Therefore two integers are a dividing pair if they have the same odd coefficient when represented in the form $(2n+1)2^m$. There are 100 odd numbers between 1 and 200, and by the pigeonhole principle a choice of 101 integers between 1 and 200 must include two with the same odd coefficient. 

\exercise{Represent the following decimals as ratios of integers.}
\begin{itemize} 
\item $0.1212121212\dots$,
\item $0.123123123123123\dots$,
\item $0.456456456\dots$.
\end{itemize}
\solution
\begin{enumerate}
\item $10^2x -x = 12 \implies x = \frac{12}{99} = 0.1212121212\dots$,
\item $10^3x-x = 123 \implies x = \frac{123}{999} = 0.123123123123123\dots$,
\item $10^3x-x = 456 \implies x = \frac{456}{999} = 0.456456456\dots$.
\end{enumerate}

\newpage
\section{Divisions}
\exercise{Find the highest common factors of the following number pairs.}
\begin{itemize}
\item 1236 and 369.
\item 144 and 900.
\item 99 and 36.
\end{itemize}

\solution
\begin{enumerate}
\item $1236 = 3\underset{1107}{\times} 369 + 129$
\item[] $369 = 2\underset{258}{\times} 129 + 111$ 
\item[] $129 = 1\underset{111}{\times} 111 + 18$ 
\item[] $111 = 6\underset{108}{\times} 18 + 3$ 
\item[] $ 18 = 6\underset{18}{\times} 3 + 0$. 
\item[] The highest common factor is 3.

\item $900 = 6\underset{864}{\times} 144 + 36$
\item[] $144 = 4\underset{144}{\times} 36 + 0$
\item[] The highest common factor is 36.

\item $99 = 2\underset{72}{\times} 36 + 27$
\item[] $36 = 1\underset{27}{\times} 27 + 9$
\item[] $27 = 3\underset{27}{\times} 9 + 0$
\item[] The highest common factor is 9.
\end{enumerate}

\exercise{Show that the highest common factor of $n$ and $n+1$ is 1.}
\solution
From Euclid's division algorithm, rewriting $m = qn + r$ in terms of $n$ and $n+1$ gives $$n+1 = 1\times n + 1$$ where $m=n+1, q=1$ and $r=1$. Then the highest common factor between $m$ and $n$ must be the same as that between $n$ and $r$. The only factor of 1 is 1, so $(n+1,n) = (n, 1) = 1$.

\exercise{Show that the highest common factor of $n$ and $n^2+1$ is 1.}
\solution
The same argument as in Exercise 4.2 applies, but this time $m=n^2+1, q=n$ and $r=1$. The highest common factor $(n^2+1, n) = (n, 1) = 1$.
\exercise{What are the possible highest common factors of $n$ and $n^2+k$ if $k<n$?}
From the equality $(n^2+k, n) = (n, k)$ the possible highest common factors of $n$ and $n^2+k$ are the factors of $n$. %If $k$ is a factor of $n$ the highest common factor $(n^2+k,n)$ is $k$, otherwise it is 1.

\newpage
\section{Contrapositive and Contradiction}
\exercise{Use proof by contradiction to show that $\sqrt 12$ is irrational.}
\solution
Assume $\sqrt12$ is a non-integer rational, then $\sqrt 12 = \frac{m}{n}$ for coprime integers $m, n$. Squaring both sides, $12 = \frac{m^2}{n^2}$. Let the prime factors of $m$ be $\{\alpha\}$ such that $m = \prod\alpha^{i_\alpha} = \alpha_1^{i_1} \alpha_2^{i_2} \alpha_3^{i_3} \dotsb$ and the factors of $n$ likewise be $\{\beta\}$. The factorization of $m^2$ is $\alpha_1^{2i_1}\alpha_2^{2i_2}\alpha_3^{2i_3}\dotsb$ so $n$ and $n^2$ have the same prime factors with different exponents. By the same logic $m$ and $m^2$ have the same prime factors. Therefore $m^2$ and $n^2$ are coprime, and the quotient $\frac{m^2}{n^2}$ is not an integer. This contradicts $12 = \frac{m^2}{n^2}$ which follows from the assumption that $\sqrt 12$ is a rational number. Therefore $\sqrt{12}$ is irrational.

\exercise{What are the contrapositives of the following statements?}
\begin{itemize}
\item Every differentiable function is continuous.
\item Every infinite set can be placed in a bijection with the rationals.
\item Every prime number is odd.
\item Every odd number is prime.
\end{itemize}
What are their converses? Which are true?\\

\solution
\begin{enumerate}
\item Every discontinuous function is not differentiable.
\item Every set that cannot be placed in a bijection with the rationals is finite.
\item Every even number is not prime.
\item Every number that is not prime is even. 
\end{enumerate}

The converse of the first statement is: every continuous function is differentiable. The converse of the second statement is: every set that can be placed in a bijection with the rationals is infinite. The converse of the third statement is the fourth statement and vice-versa. The first and third statements are true, while the second and fourth statements are false. 
 
\newpage
\section{Intersection-Enclosure and Generation}
\exercise{Check that the intersection-enclosure pattern does indeed apply to each of the examples of Sect. 6.2}
\solution
\begin{enumerate}
\item Additive subgroups by generation: \\
Let $S_0 = \{1\}$, and let $S_n$ be the set $S_{n-1} \cup \{x-y\mid x,y \in S_{n-1}\}$ generated by taking the difference of the elements of $S_{n-1}$ for $n \geq 1$. For example, $S_1 = \{0, 1\}$ and $S_2 = \{-1, 0, 1\}$. Let $S_\infty = \bigcup_n S_n$. If $x, y \in S_\infty$ then for some $i, j$ we have $x\in S_i$ and $y\in S_j$. If $k = \text{max}\{i, j\}$ then $S_k$ contains both $x$ and $y$, and $S_{k+1} \subset S_\infty$ contains $x-y$. Therefore $S_\infty$ is an additive subgroup. Let $B$ be the smallest subgroup containing 1. $S_\infty$ is an additive subgroup containing 1, so $S_\infty$ contains $B$. Any additive subgroup containing 1 must contain $S_k$ for all $k$ because the elements of $S_k$ are generated only by the number 1 and the additive subgroup condition, so $B$ contains $S_\infty$. Therefore $S_\infty=B$ is the smallest additive subgroup generated by 1.
\item[] Additive subgroups by intersection-enclosure: \\
Let $\{S_\alpha\}_{\alpha \in I}$ be the set of all additive subgroups containing 1. The intersection $S=\bigcap_{\alpha\in I} S_\alpha$ contains 1 because every $S_\alpha$ contains 1. If $x, y \in S$ then $x,y \in S_\alpha$ for every $\alpha \in I$. By definition $x-y \in S_\alpha$ for every $\alpha \in I$. Therefore $x-y \in S$ and $S$ is an additive subgroup. Any additive subgroup containing $1$ contains $S$, so $S$ is the smallest additive subgroup containing $1$. 

\item Vector sub-spaces \\
A non-empty subset $B$ of $\mathbb{R}^n$ is a \textit{vector sub-space} if the sum of any two elements of $B$ is also in $B$, and any scalar multiple of an element of $B$ is also in $B$. Let $\{S_\alpha\}_{\alpha \in I}$ be the set of vector spaces containing a set of vectors $A$. Let $S=\bigcap_{\alpha \in I} S_\alpha$ be the intersection of those sets. If $x, y \in S$ then $x,y\in S_alpha$ for every $alpha$ in $I$, so $x+y$ is in every $S_\alpha$ and $x+y$ is in $S$. If $x \in S_\alpha$ for every $\alpha \in I$ then $\lambda x \in S_\alpha, \alpha \in I$ and $\lambda x \in S$. Therefore $S$ satisfies the vector sub-space conditions. 

\item Closure \\
A subset $B$ of $\mathbb{R}$ is said to be \textit{closed} if the limit of every convergent sequence in $B$ is also in $B$. Let $S$ be the intersection of every closed set containing a subset $A$ of $\mathbb{R}$. If the sequence $x_n \in S \forall n$, then $x_n$ is in every closed set $C_A$ containing $A$. If $x_n \to x \in \mathbb{R}$ converges, then its limit $x \in C_A$ by the definition of closed. As the intersection of every $C_A$, $S$ too contains $x$. Therefore $S$ is closed. 

\item Convexity \\
A subset, $B$, of $\mathbb{R}^n$ is said to be \textit{convex} if the straight line between any two of its points is also contained in it. Let $S$ be the intersection of every convex set containing a subset $A$ of $\mathbb{R}^n$ and suppose $x, y \in S$. Then $x, y$ are in every convex set containing $A$, and by the definition of convexity the straight line between $x$ and $y$ is in every convex set containing $A$. Therefore the straight line $\lambda x + (1- \lambda y), \lambda \in (0, 1)$ between $x$ and $y$ is in $S$, and $S$ is the \textit{convex hull}, the smallest convex set containing $A$.  

\end{enumerate}

\exercise{Given 3 points in the plane, what is their convex hull? Distinguish according to whether the 3 points are collinear.}
\solution 
If the three points $x, y, z\in \mathbb{R}^2$ are collinear, their convex hull is a line. Otherwise, the convex hull consists of the points bounded by a triangle. 

\exercise{Given a subset of the plane, how many steps are required for the generation algorithm for convexity to terminate?}
\solution
Let $A_0\subset \mathbb{R}^n$ be a subset of the plane, and $A_{n+1}$ be the set $A_n \cup \{\lambda x + (1-\lambda)y \mid x, y \in A_{n}, \lambda \in (0,1)\}$. The question is under what conditions does $$A_k = A_{k+1}.$$ Denote the convex hull of $A_0$ as $A'$. Suppose $A_0$ is one point. Then $A'$ = $A_0$ and the generation algorithm terminates after zero steps. If $A_0$ is two points, then $A_1$ is the set $A_0 \cup \{\lambda x + (1-\lambda)y \mid x, y \in A_0, \lambda \in (0,1)\} =A'$, the straight line between the two points in $A_0$, and the generation algorithm terminates after one step. If $|A_0| \geq 3$ and the elements of $A_0$ are not collinear, then the first step of the generation algorithm $A_1$ consists of the boundary of $A_0$ and lines connecting the points of $A_0$. The second step of the generation algorithm is then the convex hull $A'$, the lines connecting all points on the boundary of $A_0$. In general, for a subset of the plane with three or more non collinear points, the generation algorithm for convexity terminates after two steps. 

\exercise{A subset of $\mathbb{R}$ is said to be binarily division invariant if dividing any element by 2 results in an element of the subset. Check that intersection-enclosure applies. Also analyze generation for this property and show that it terminates with $A_\infty$.}
\solution
\begin{enumerate}
\item Check that intersection-enclosure applies. \\
Let $A$ is a subset of $\mathbb{R}$ and let $B$ be the intersection of every binarily division invariant subset containing $A$. If $x\in B$ then $x$ is in every binarily division invariant subset $A_\beta$ containing $A$, so $x/2$ is in every $A_\beta$ and $x/2 \in B$. Therefore binary division invariance holds under the intersection-enclosure pattern.  
\item Analyze generation for this property. \\
Let $A_0$ be a subset of $\mathbb{R}$ and let $A_{n+1} = A_n \cup \{x/2 \mid x \in A_n\}$. Then the set $A_\infty = \bigcup_n A_n$ is the result of the generation algorithm. If $x \in A$ then $x \in A_k$ for some $k$ and $x/2 \in A_{k+1} \subset A_\infty$. Therefore $A_\infty$ is binarily division invariant and the generation algorithm terminates at $A_\infty$.
\end{enumerate}

\exercise{A subset of $\mathbb{R}$ is said to be binarily division invariant and zero-happy if dividing any element by 2 results in an element of the subset, it is closed, and, in addition, if it contains zero then it also contains 2. Check that intersection-enclosure applies. Also analyze generation for this property and show that it does not always terminate with $A_\infty$.}
\solution 
\begin{enumerate}
\item Check that intersection-enclosure applies. \\
Checking intersection-enclosure for closure was done in Exercise 6.1, and in Exercise 6.4 for binary division invariance. All that remains is to check the third property—if the set contains zero then it also contains 2. Let $S$ be the intersection of all binarily division invariant and zero happy sets containing a subset $A$ of $\mathbb{R}$. If $2\in S$ then 2 is in every such set containing $A$, and by definition all such sets contain zero as well. As $S$ is the intersection of those binarily division invariant and zero happy sets containing $A$, $S$ also contains zero. Therefore intersection-enclosure applies to all of the conditions for binarily division invariant and zero happy sets.

\item Analyze generation for this property. \\
Let $A_0$ be a subset of $\mathbb{R}$ and let $A_{n+1}$ be the set $$A_n \cup \{x/2 \mid x\in A_n\} \cup \{ x_n \to x \mid x_n \in A_n \text{ and } x \in \mathbb{R} \}$$ 
with the additional property that $A_{n+1} = A_{n+1} \cup \{2\}$ if $0 \in A_n$. Let $A_\infty = \bigcup_n A_n$. All that remains is to check that the properties hold for $A_\infty$. Let $x$ be an element of $A_\infty$. Then $x\in A_n$ for some $A_n$ and $x/2 \in A_{n+1} \subset A_\infty$, so the first property holds. If $0\in A_\infty$ then $0 \in A_n$ for some $n$ so that $2 \in A_{n+1} \subset A_\infty$ and the third property holds. The closure condition requires that the limit of every convergent sequence in $A_\infty$ is contained in $A_\infty$, but the limit of a sequence such that $x_n \in A_n$ will not be in $A_\infty$. For example, the output $A_\infty$ of the generation algorithm for the set $A_0 = \{1\}$ will contain arbitrarily small binary divisions of $1$, but the limit of the convergent sequence $\frac{1}{2^n}\to 0$ is not contained in any $A_n$. In such cases the generation algorithm does not terminate at infinity and must be repeated. 

\end{enumerate}

\exercise{A subset of $\mathbb{R}$ is said to be \textit{open} if its complement is closed. Will there be a smallest open subset containing $[0,1]$?}
\solution 
Let $\{S_\alpha\}_{\alpha \in I}$ be the set of open subsets containing $[0,1]$ and let $S$ be the intersection $\bigcap_{\alpha \in I} S_\alpha$. Let $x_n$ be a sequence such that $x_n \notin S$. The openness property is satisfied if the limit of the sequence $x_n \to x \notin S$ for every such sequence. Consider the sequence $a_n = -\frac{1}{n}$. For every $a_n$ there is an open subset $S_n = [-\frac{1}{n+1}, 1]$ such that $S_n$ is an open subset containing $[0,1]$ and $a_n \notin S_n$. Since $S$ is the intersection of all such open subsets, $a_n \notin S$. But $S$ contains the limit of $a_n$ which is 0. Therefore the smallest open subset containing $[0,1]$ does not exist. 

\exercise{Show that if $p$ is a polynomial with real coefficients and $p(z) = 0$ then $p(\bar{z})$ is also zero. Use this fact in conjunction with the fundamental theorem of algebra to show that every real polynomial of odd order has a real zero.}
\solution Let $$p(x) = a_nx^n + a_{n-1}x^n-1 + \dotsb + a_1x + a_0$$ with all the $a_i$'s real and suppose $p(z) = 0$. Then if $z=a+bi$,  we have $$p(z) = a_n(a+bi)^n + a_{n-1}(a+bi)^{n-1} + \dotsb + a_1(a + bi) + a_0= 0.$$ This expression simplifies to a real expression in terms of powers of $a$, even powers of $b$ and the $a_i$'s, plus a complex term with a coefficient in terms of powers of $a$, odd powers of $b$ and the $a_i$'s. Since $z=a+bi$ is a zero, both the real expression and the coefficient of $i$ evaluate to zero. The conjugate $\bar z =a-bi$ can be interpreted as either taking the negative of $b$ or the negative of $i$. In the first interpretation, the real expression is unchanged because it only involves even powers of $b$. In the second interpretation, if the coefficient of $i$ evaluates to zero then replacing $i$ with $-i$ still results in a zero complex part. Therefore if $z$ is a zero of a real polynomial $p$ then its conjugate $\bar z$ is a zero as well. 

\newpage
\section{Differences of Invariants}
\exercise{What sizes of board $m\times n$, can be covered by $2\times 2$ squares?}
\solution 
Boards of size $m\times n$ can be covered by $2\times 2$ squares if and only if $2\mid m$ and $2\mid n$. Trivially a $2\times 2$ square can cover a board such that $m=2$ and $n=2$. If $m=2$ and $n=2k$, then the board can be covered by a single row consisting of $k$ copies of a $2\times 2$ square. If $m=2l$ and $n=2k$ then the board can be covered by $l$ copies of a $2\times 2k$ board. Therefore if a board of size $m\times n$ such that $2 \mid m$ and $2 \mid n$ then it can be covered by $2\times 2$ squares. \\
\indent To show the converse, if a board of size $m\times n$ that can be covered by $2\times 2$ squares then so can a board of size $m-2\times n-2$ provided $m$ and $n$ are greater than 2. A $1\times n$ or $m\times 1$ board cannot be covered by $2\times 2$ squares, so any $m\times n$ board such that $m-2k =1$ or $n-2k = 1$ cannot be covered. These are precisely the boards with odd $m$ or $n$. Therefore if a $m\times n$ board can be covered by $2\times 2$ squares then $m$ and $n$ are both even. 

\exercise{Define multiplication on $\mathbb{Z}_k$ by taking $xy$ and then taking its remainder on division by $k$. For what values of $x$ and $k$ does there exist a number $y$ such that $xy = 1$?}
\solution 
If the remainder of $xy$ divided by $k$ is 1, then $xy +qk = 1$. Given $x$ and $k$ there exist integers $y$ and $q$ satisfying this equation if and only if 1 is the greatest common factor of $x$ and $k$. Therefore there exists a number $y$ such that $xy=1$ on $\mathbb{Z}_k$ as defined if $x$ and $k$ are coprime. 
\exercise{Let $n$ be a positive integer. Let $f(n)$ denote the sum of its digits. Show that the remainder on division by 9 is invariant under passing from $n$ to $f(n)$. Use this to show that if we keep summing up the digits of $n$ until we get a number less than 10 then $n$ is divisible by 9 if and only if this last number is 9. Repeat and reformulate the result for 3. More generally, does this approach work for any other numbers? What if we change our number base?}
\solution
The base 10 representation of a positive integer $n$ can be expressed by $$n = n_k10^k + n_{k-1}10^{k-1} + \dotsb + n_1 10 + n_0$$ where the digits of $n$ are $n_k n_{k-1}\dots n_1 n_0$. In terms of this expression, we can write $f(n)= n_0 + n_1 + \dotsb + n_k$. Using the fact that $10 \equiv 1$ (mod 9), we have $$n \equiv n_k1^k + n_{k-1}1^{k-1} + \dotsb + n_0 \equiv n_k + n_{k-1} + \dotsb + n_0 \equiv f(n) \text{ (mod 9)}$$ 
which shows that divisibility by 9 is invariant between $n$ and $f(n)$. By repeatedly passing from $n$ to $f(n)$ one can reduce $n$ to a single digit number. The only non-zero single digit number divisible by nine is 9, so $n$ is divisible by 9 if and only if the termination of this process results in 9. \\
\indent Because $10 \equiv 1$ (mod 3) the same result holds for 3. If the base 10 representation of $n$ is congruent to zero modulo 3, so is the sum of its digits $f(n)$. At the termination of the procedure, if the single digit output $f^k(n)$ is either 3, 6 or 9 then we can conclude that 3 divides $n$. \\
\indent In general, given a positive integer in a base $b$, divisibility by $d$ is invariant under taking the sum of the digits if $b \equiv 1$ (mod $d$). Only under those conditions does this process work.

\newpage
\section{Linear Dependence, Fields and Transcendence}
\exercise{Show that if $y\in \mathbb{C}$ is algebraic and $x^k = y$ then $x$ is also algebraic.}
\solution
If $y\in \mathbb{C}$ is algebraic then for some polynomial $p$, $p(y) = 0$. Let $q$ be the polynomial $p(y^k)$. Then $q(x)=p(x^k)=p(y)=0$ and $x$ is also algebraic.
\exercise{Show that if $y$ is transcendental then $y^k$ is transcendental for all counting numbers $k$. Show also that $y^{1/k}$ is transcendental.}
\solution
It is sufficient to prove the contrapositive, that if $y^k$ is algebraic then $y$ is algebraic. Exercise 8.1 showed that if $y$ is algebraic and $y=x^k$ then $x$ is algebraic. Therefore for transcendental $y$, $y^k$ is transcendental for all counting numbers $k$. \\
\indent Algebraic numbers are closed under multiplication. If $x$ is algebraic, then so is $x\cdot x = x^2$. If $x^n$ is algebraic, then $x\cdot x^n = x^{n+1}$ is algebraic. By induction, if $x$ is algebraic then for all counting numbers $k$, $x^k$ is algebraic. In other words, $y^{1/k}$ is algebraic implies ${y^{(1/k)}}^k = y^{(1/k)k} = y$ is algebraic. Therefore the contrapositive is true and if $y$ is transcendental then $y^{1/k}$ is transcendental.  \\
\indent \textbf{Alternate proof: }
If $y^{1/k}$ is transcendental then the set $$S_1 = \{y^{j/k} \mid j = 0, \dots, n\}$$ is linearly independent over $\mathbb{Q}$ for all counting numbers $n$. Any subset of a linearly independent set is also linearly independent, so the set $$S_2 = \{y^{j/k} \mid j = 0, k, 2k, \dots, nk\} \subset S_1$$ is linearly independent, and there are no linear combinations of powers of $y$ with rational coefficients that are equal to zero. Equivalently, $y$ is transcendental.


\exercise{Show that the smallest sub-field of $\mathbb{R}$ which is closed under the taking of the powers 1/2, 1/3, 1/5 is contained in the set of algebraic numbers.}
\solution
Let $\{S_\alpha\}_{\alpha\in I}$ be the set of subfields of $\mathbb{R}$ closed under the taking of the powers 1/2, 1/3 and 1/5 indexed by an arbitrary index set $I$. The smallest set satisfying the above properties is the intersection $\bigcap_{\alpha\in I} S_\alpha$ of all such sets. Any set satisfying the above properties therefore contains the smallest set satisfying those properties. A real number $y$ is algebraic if there is a nonzero rational polynomial $p$ such that $p(y) = 0$. If such a polynomial $p$ exists then a polynomial $q$ exists for $x = y^{1/2}$ with the definition $q(z) =  p(z^2)$ such that $q(x)=p(x^2)=p(y)=0$. Likewise if $x=y^{1/3}$ or $x=y^{1/5}$ then $q(z)=p(z^3)$ and $q(z)=p(z^5)$ respectively. Therefore algebraic numbers are closed under the taking of the power 1/2, 1/3, 1/5 and the smallest such set is contained in the set of algebraic numbers. 

\newpage
\section{Formal Equivalence}
\exercise{Using a ruler and compass construction, duplicate the square.}
\solution
To duplicate a square means given a square of side length $x$, construct a square of area $2x^2$. Let the points $ABCD$ form a square. Centered at point $A$, draw a circle of radius $AB$ and extend the line $AB$ through the circle. Let $E$ be the intersection of this line with the circle. Also extend the line $AD$ through the circle and call the intersection $F$. Draw lines $BD$, $DE$, $EF$ and $FB$. The points $BDEF$ form a square. Each side is the length $BD$, the diagonal of the square $ABCD$. If each side of the original square is length $x$, then the length of $BD$ is $x\sqrt 2$ by the Pythagorean theorem. Therefore the area of $BDEF$ is $2x^2$ as desired.

\exercise{Show directly that if $a, b, r$ are rationals with $r > 0$, then $a + b\sqrt r$ satisfies a polynomial with rational coefficients.}
\solution
Let $p(x) = x^2 - 2ax - (b^2r -a^2)$. Then
\begin{align*}
p(a + b\sqrt r) &= (a + b\sqrt r)^2 - 2a(a + b\sqrt r) - (b^2r - a^2)  \\
&= a^2 + 2ab\sqrt r + b^2r - 2a^2 - 2ab\sqrt r - b^2r + a^2 \\
&= 0
\end{align*}

\newpage
\section{Equivalence Extension}
\exercise{Prove that if $k$ is a positive integer and $x$ is a positive real number then $x$ has a $k$th root in the real numbers.}
\solution
Let $E$ be the set $E = \{y \mid y^k \leq x\}$. This set contains 0 so it is non-empty, and it is bounded above by the maximum of 1 and $x$. The real numbers observe the least upper bound property, so the set $E$ has a least upper bound $l$. It is necessary to show that $l = \sqrt[k]x$. \\
\indent Suppose $l^k < x$ and let the sequence 
$$l_n = l + \frac{1}{n}$$
so that $l_n > l$ for all $n$. The sequence $\{l_n\}$ converges to $l$, so there is an $N$ such that $l_n$ is arbitrarily close to $l$ when $n\geq N$. Choose $N$ such that $l_n^k-l^k < x -l^k$ when $n\geq N$. This implies $l_n^k < x$ and $l_n \in E$. Thus $l_n$ is an element of $E$ greater than $l$ which contradicts the assumption that $l$ is an upper bound. Then we can conclude $l^k\geq x$. \\
\indent Now suppose $l^k > x$. The sequence
$$l_n = l - \frac{1}{n}$$
which also converges to $l$ poses the same problem. For all $n$ we have $l_n < l$ but there exists $n$ such that $l_n^k > x$. Then $l_n$ is an upper bound of the set $E$ greater than $l$, which contradicts the assumption that $l$ is the least upper bound. We conclude that $l^k = x$ and $l$ is the $k$th root of $x$.

\exercise{Prove that every subset of the reals that is bounded below has a greatest lower bound.}
\solution
Let $E$ be a subset of the real numbers that is bounded below and let $L$ be the set of lower bounds of $E$, the set 
$$L := \{l : l \leq x, \forall x \in E\}.$$
The set $L$ is nonempty since $E$ is bounded below, and $L$ is bounded above by any $x$ in $E$. Let $u$ be the least upper bound of $L$ in the real numbers guaranteed by the least upper bound property. We show that $u$ is a lower bound of $E$ and conclude that it is the greatest lower bound. \\
\indent Suppose $u\notin L$, $u$ is not a lower bound of $E$. Then there exists $x$ in $E$ such that $x < u$. For every lower bound $l$ of $E$ we have $l\leq x$. Then $x$ is an upper bound of $L$. The existence of an upper bound of $L$ less than $u$ contradicts the assumption that $u$ is the least upper bound of $L$. So we conclude $u \in L$.\\
\indent We have by definition of the least upper bound that $u\geq l$ for all $l\in L$. Thus $u$ is a lower bound of $E$ and it is the greatest lower bound of $E$. 

\exercise{Construct the positive rationals directly from the natural numbers, and then construct the negative rationals with them. Establish a bijection between the rationals constructed this way and the rationals constructed the original way. Ensure that the bijection is the identity on the natural numbers and that it commutes with multiplication and division.}
\solution 
Let $S_1$ be the set $\mathbb{N} \times (\mathbb{N} - \{0\})$ and define an equivalence relation $\sim$ on $S$ such that $ (p_1, q_1) \sim (p_2, q_2)$ if $p_1q_2 = p_2q_1$. The fact that the equivalence relation is reflexive and symmetric follows from those properties of the equality relation. We will show that $\sim$ is transitive as well. If
$$(p_1, q_1)\sim(p_2, q_2) \text{ and } (p_2, q_2) \sim (p_3, q_3)$$
then
$$p_1q_2 = p_2q_1 \text{ and } p_2q_3 = p_3q_2.$$
So
$$q_2 = p_2q_1/p_1 \text{ and } q_2 = p_2q_3/p_3$$
and this division is well defined since it has an answer in $\mathbb{N}$. This implies
$$p_1p_2q_3 = p_3p_2q_1.$$
If $p_2 = 0$ then $p_1q_2 = p_2q_1 = 0$. Since $q_1 \in \mathbb{N} - \{0\}$ it cannot be zero. Therefore $p_1 = 0.$ Similarly $p_3q_2 = p_2q_3 = 0$ and $p_3 = 0$. We have that $p_1q_3 = p_3q_1 = 0$ and conclude $(p_1, q_1) \sim (p_3, q_3)$. \\
\indent If $p_2$ is not zero, then the division 
$$p_1p_2q_3/p_2 = p_3p_2q_1/p_2$$
is defined in $\mathbb{N}$ and we have 
$$p_1q_3 = p_3q_1$$ 
so that $(p_1, q_1) \sim (p_3,  q_3)$ as desired, and $\sim$ is an equivalence relation.\\ 
\indent We call the set of equivalence classes of $\sim$ on $\mathbb{N} \times (\mathbb{N} - \{0\}$) the nonnegative rational numbers, $Q^+$. Let $(p, q)$ be the representative element of each equivalence class $[(p,q)]$ such that $p$ and $q$ are co-prime. If some $p$ and $q$ have a common factor $u$ then $[(p/u, q/u)] = [(p,q)]$ so the choice of representative is well defined. \\
\indent Next we define addition and multiplication on $Q^+$. We define 
$$[(p_1, q_1)] \cdot [(p_2, q_2)] = [(p_1p_2, q_1q_2)]$$
and show that $\cdot$ is well defined. That is, the product of two equivalence classes must be consistent across any choice of representative. We show that
$$(p_1, q_1) \sim (\tilde p_1, \tilde q_1), (p_2,  q_2) \sim (\tilde p_2, \tilde q_2) \implies (p_1p_2, q_1q_2) \sim (\tilde p_1\tilde p_2, \tilde q_1 \tilde q_2).$$ 
The definition of the equivalence relation gives 
$$p_1p_2\tilde q_1\tilde q_2 = (p_1\tilde q_1)(p_2\tilde q_2) = (\tilde p_1 q_1) (\tilde p_2 q_2) = \tilde p_1 \tilde p_2 q_1 q_2$$
which proves the result. \\
\indent We define addition on $Q^+$ by 
$$[(p_1, q)] + [(p_2, q)] = [(p_1 + p_2 , q)]$$
and show that addition is well defined for any choice of $q$. First, there always exists such a $q$ for if $x = [(p_1, q_1)]$ and $y=[(p_2, q_2)]$ then we can choose the representation
$$x = [(p_1q_2, q_1q_2)], y =[(p_2q_1, q_1q_2)]$$
such that $q = q_1q_2$ and addition is defined for this $q$. Consider $q'$, the least common multiple of $q_1$ and $q_2$. For any other choice of $q$ we have $q =r q'$. If we have
$$x = [(s_1, q')], y = [(s_2, q')]$$ 
with the least common multiple in the second coordinate, then then sum is
$$ x+ y =[(s_1 + s_2, q')].$$ 
For any other choice of $q$ we have
$$[(rs_1, rq')] + [(rs_2, rq')] = [(r(s_1 + s_2), rq')]$$
and the fact that
$$(r(s_1 +s_2), rq') \sim (s_1 + s_2, q')$$
shows that the representations agree. Order is defined on $Q^+$ by $[(p_1,q)]>[(p_2,q)]$ if $p_1 > p_2$. Every pair of equivalence classes has a representation such that the second coordinates agree, so the relation is well defined. Note that, as in $\mathbb{N}$, subtraction sum $x-y$ is only defined on $Q^+$ if $x\geq y$. \\
\indent Let $S_2$ be the set $Q^+ \times Q^+$ and define the equivalence relation $\sim$ on $S_2$ such that $(x_1, y_1) \sim (x_2, y_2)$ if $x_1 + y_2 = x_2 + y_1$. As before, reflexivity and symmetry follow from those properties of the $=$ relation. For transitivity, if $(x_1,  y_1) \sim (x_2, y_2)$ and $(x_2, y_2) \sim (x_3, y_3)$ then
$$y_2 = x_2 + y_1 - x_1 \text{ and } y_2 = x_2 + y_3 - x_3$$
and the subtraction is well defined because it has an answer in $\mathbb{Q}$, which is $y_2$. This gives 
$$x_2 + y_1 - x_1 = x_2 + y_3 - x_3$$ 
and rearranging terms yields
$$x_3 +y_1 = x_1 + y_3$$
which is the equivalence condition $(x_1, y_1) \sim (x_3, y_3)$ and thus $\sim$ is transitive.\\
\indent The equivalence classes are the pairs of rational numbers $(x, y)$ with the same sum $x-y$. Let $Q$ be the set of equivalence classes of $S_2$ under the equivalence relation $\sim$. Each equivalence class can be distinguished by a unique member in which at least one entry is zero. For if $x_1 > y_1$ then $x_1-y_1$ is defined on $Q^+$ and we can write
$$(x_1 - y_1) + y_1 = x_1 + 0$$
which is the equivalence condition for
$$(x_1 - y_1, 0) \sim (x_1, y_1).$$
Likewise if $y_1>x_1$ then $y_1-x_1$ is defined on $Q^+$ and 
$$0 + y_1 = x_1 + (y_1 - x_1)$$
which shows that
$$(0, y_1-x_1) \sim (x_1, y_1).$$
In the case that $x_1 =  y_1$, we have
$$0 + y_1 = x_1 + 0$$
so that
$$(0,0) \sim (x_1, y_1).$$
We define addition on $Q$ by addition of the representative elements:
\begin{align*}
[(x,0)] + [(y,0)] &= [(x+y,0)] \\
[(x,0)] + [(0, y)] &= [(x, y)] \\
[(0,x)] + [(0, y)] &= [(0, x+y)]. 
\end{align*}
We also define multiplication in terms of the representative elements:
\begin{align*}
[(x, 0)] \cdot [(y, 0)] &= [(xy, 0)] \\
[(x,0)] \cdot [(0, y)] &= [(0, xy)] \\
[(0,x)] \cdot [(0, y)] &= [(xy, 0)]. 
\end{align*}
\indent Let the set $\mathbb{Q}$ be the rationals constructed the original way. We will show that there is a bijection between $Q$ and $\mathbb{Q}$ that is compatible with multiplication and division. Let $j\colon Q^+\to \mathbb{Q}$ be such that 
$$j\colon [(p,q)] \mapsto [(i(p),i(q))]$$
where $i$ maps the natural number $p$ to the integer $(p, 0)$ defined in the original way. Let $k\colon Q\to \mathbb{Q}$ be the map
\begin{align*}
(x,0) &\mapsto j(x) \\
k\colon (0,x) &\mapsto -j(x) \\
(0,0) &\mapsto j(0) 
\end{align*}
such that $k$ is defined on the representative elements of $Q$. If $n$ is a natural number in $Q$ then it is in the equivalence class represented by $([(n, 1)], 0)$ and we have
$$k((n,1),0) = j(n,1) = i(n) - i(0) = n$$ 
as desired. \\
\indent In the case of multiplication and division it is necessary to check that $k$ commutes with those operations for every pair of representative elements:

$$k((p_1,q_1),0) \cdot k((p_2,q_2),0) = j(p_1,q_1)\cdot j(p_2,q_2) = (i(p_1), i(q_1)) \cdot (i(p_2), i(q_2))$$
which is the product of two elements of $\mathbb{Q+}$ and is defined by
$$[(p_1, q_1)] \cdot [(p_2, q_2)] = [(p_1p_2, q_1q_2)].$$
Then we have 
$$(i(p_1), i(q_1)) \cdot (i(p_2), i(q_2))= (i(p_1p_2), i(q_1q_2)) = j(p_1p_2,q_1q_2) = k((p_1p_2,q_1q_2),0)$$ 
which is 
$$k(((p_1,q_1),0)\cdot ((p_2,q_2),0))$$
as desired. \\
\indent Multiplicative inverses on $Q$ are defined in terms of the representative elements of its equivalence classes, such that
\begin{align*}
((p,q),0)^{-1} &= ((q,p),0) \text{ and}\\
(0,(p,q))^{-1} &= (0, (q,p)).
\end{align*}
Then under composition with $k$,
$$k(((p,q),0)^{-1}) = k((q,p), 0) = j(q,p) = (i(q),i(p)) = (i(p),i(q))^{-1} = j(p,q)^{-1} = k((p,q),0)^{-1}$$
and division commutes with the bijection.

\exercise{Give examples of subsets of the reals that do not have greatest lower bounds.}
\solution Any subset of the reals that is bounded below has a greatest lower bound. Therefore the sets that do not have greatest lower bounds are not bounded below. Examples of such sets are:
\begin{align*}
\{x &\mid x < 0\} \\ 
\{x &\mid x \leq 0\} \\ 
\{x &\mid x = (-1)^nn^2 \text{ for } n\in \mathbb{N}\}
\end{align*}
\exercise{Show that a monotone decreasing sequence of real numbers is bounded below if and only if it is convergent.}
\solution Let $\{a_n\}$ be a monotone decreasing sequence in $\mathbb{R}$. Suppose it is bounded below, and let $L$ be the set of lower bounds. The least upper bound property dictates that the supremum $l$ of $L$ is in $\mathbb{R}$. Given $\epsilon >0$, we have that $l+\epsilon$ is not a lower bound of $\{a_n\}$. Then there is a $N$ such that $a_N < l+\epsilon$. Since $\{a_n\}$ is monotone decreasing, if $n>N$ then  $a_n-l < \epsilon$. Therefore the sequence $\{a_n\}$ is convergent and it converges to $l$. \\
\indent Let $\{a_n\}$ be a monotone decreasing sequence in $\mathbb{R}$, and suppose it is not bounded below. Given $l$ and $\epsilon$, there is an $a_N$ such that $a_N < l-\epsilon$. Because $a_n$ is monotone decreasing, for $n>N$ we have that $a_n < a_N < l-\epsilon$ and by rearranging we have $l - a_n > \epsilon$. Therefore $a_n$ cannot converge to $l$. 

\newpage

\section{Proof by Classification}
\exercise{Find all Pythagorean triples with one side equal to 12.} 
\solution
These are positive integer triples $(k, u, v)$ such that one of
\begin{align}
k(v^2 - u^2) &= 12\\
2kuv &= 12 \\
k(u^2 + v^2) &= 12
\end{align}
is true, with $u$ and $v$ co-prime, $u+v$ odd and $u<v$. Looking at the first equation, we can see that 
$$k(v^2 - u^2) = 12$$
if and only if $v^2-u^2$ divides 12. We have $u+v \leq 12$ if $v^2-u^2 = (v-u)(v+u) \leq 12$. There are 14 co-prime pairs $u,v$ such that their sum is odd and less than or equal to 12: (1, 2) (1,4) (1,6) (1,8) (1,10) (2,3) (2,5) (2,7) (2,9) (3,4) (3,8) (4,5) (4,7) and (5,6). Of these, there are 3 such that their sum divides 12: (1,2) (1,5) (1,11). However, only (1,2) is such that $v^2-u^2$ divides 12. Then the first equation gives (1,2) as a generating pair. That triple is:
$$(4,1,2) \mapsto (12, 16, 20).$$
\indent The second equation $2kuv = 12$ requires co-prime pairs $(u,v), u<v, u+v$ odd such that $uv$ divides 6, that is $uv= 1,2,3$ or 6. Valid pairs are (1,2) (2,3) and (1,6). These generate the following Pythagorean triples:
\begin{align*}
(3,1,2) &\mapsto (9, 12, 15) \\
(1,2,3) &\mapsto (5, 12, 13) \\
(1,1,6) &\mapsto (35, 12, 37).
\end{align*}
\indent The third equation $k(u^2 + v^2) = 12$ specifies the length of the hypotenuse. Candidate pairs will be those such that $u$ and $v$ are positive integers with squares less than 12. Integers of this form are 1, 2 and 3 which generate (1,2) and (2,3) as candidate pairs. However, we have $1^2 + 2^2 = 5$ and $2^2 + 3^2 = 13$ neither of which divide 12. Therefore there are no Pythagorean triples with a hypotenuse of length 12. \\
\indent In summary there are four Pythagorean triples with a side of length 12, and they are (12, 16, 20) (9, 12, 15) (5, 12, 13) and (35, 12, 37). \vspace{16pt}\\
\indent Another way to look at this is to look at generating triples with $k=1$ such one of the three sides is less than 12. These are:
\begin{align*}
(1,1,2) &\mapsto (3,4,5) \\
(1,1,4) &\mapsto (15,8,17) \\
(1,1,6) &\mapsto (35,12,37) \\
(1,2,3) &\mapsto (5,12,13) \\
(1,3,4) &\mapsto (7, 24, 25).
\end{align*}
\indent By varying $k$ on those triples with side length that divides 12, we have
\begin{align*}
(3,1,2) &\mapsto (9, 12, 15) \\
(4,1,2) &\mapsto (12,16,20)\\
(1,1,6) &\mapsto (35,12,37) \\
(1,2,3) &\mapsto (5,12,13)
\end{align*}
which are the same triples as before.

\exercise{Suppose $a,b,c,k$ are positive integers and $$c^k = ab,$$ with $a,b$ co-prime. Does it follow that $a$ and $b$ are the $k$th powers of positive integers?}
\solution
In terms of prime factors, we can write
$$c = c_1^{\phi_1} \cdot c_2^{\phi_2} \cdot c_3^{\phi_3} \cdots c_n^{\phi_n}$$
and to the $k$ power
$$c^k = c_1^{k\phi_1}\cdot c_2^{k\phi_2}\cdot c_3^{k\phi_3}\cdots c_n^{k\phi_n}.$$
\indent Since $ab = c^k$, $a$ and $b$ are each the product of prime factors $c_i$ of $c$ such that, if $c_i^{\beta_i}$ is the greatest power of $c_i$ that divides $b$, then $c_i^{k\phi_i - \beta_i}$ divides $a$ and is the greatest power of $c_i$ that divides $a$. This is to ensure that $c_i^{k\phi_i - \beta_i}c_i^{\beta_i} = c_i^{k\phi_i}$ divides $c^k =ab$. So we can write 
\begin{align*}
a &= c_1^{k\phi_1-\beta_1}\cdots c_n^{k\phi_n -\beta_n} \\
b &= c_1^{\beta_1}\cdots c_n^{\beta_n}.
\end{align*}
for $0 \leq \beta_i \leq k\phi_i$. Since $a$ and $b$ are co-prime, there is no factor $c_i$ such that $c_i$ divides both $a$ and $b$. If $0 < \beta_j<k\phi_j$ for any $j$ then $\beta_j$ divides both $a$ and $b$. Therefore either $\beta_i= 0$ or $\beta_i = k\phi_i$ for all $i$. Since every prime factor of $a$ and $b$ has a power that is a multiple of $k$, this implies that $a$ and $b$ are the $k$th power of positive integers.

\exercise{Suppose we want to find all rational Pythagorean triples, that is right-angled triangles with all sides rational. Can we classify these?}
\solution
For any rational Pythagorean triple $(x, y, z)$ we can multiply by the greatest common denominator to get an integer Pythagorean triple, $(kx, ky, kz)$. Inverting this operation, we have that every rational Pythagorean triple is of the form $(a/k, b/k, c/k)$ where $k$ is rational and $(a, b, c)$ is an integer Pythagorean triple. Then the rational Pythagorean triples are also classified by positive integer triples $(k, u, v)$ where $u<v$, $u+v$ odd, and $u, v$ co-prime; however, rational triples are generated by rational $k$. 

\exercise{Consider the map from Pythagorean triples to the length of the longest side as a natural number. Is this map surjective? Is it injective? What about for the other two sides?}
\solution
The length of the longest side of a right triangle described by a Pythagorean triple $(a,b,c)$ is the length of the hypotenuse $c$. We have that a Pythagorean triple $(a, b, c)$ can take on values $c$ for its longest side such that 
$$c = k(u^2 + v^2)$$
for positive integer $k$ and co-prime $u, v$ such that $u + v$ is odd and $u < v$. The map from a Pythagorean triple to the length of its longest side, $h \colon (a,b,c) \mapsto c$ is therefore the composition of the map 
$$g\colon (k, u, v) \mapsto k(u^2 + v^2)$$
and the bijection 
$$f\colon (a,b,c) \mapsto \left(\text{gcd}(a,b,c), \frac{\sqrt{(c-a)/2)}}{\text{gcd}(a,b,c)} , \frac{\sqrt{(c+a)/2}}{\text{gcd}(a,b,c)}\right)$$
i.e. the map $h = g \circ f$. So the possible values of $c$ are in the range of $k(u^2 + v^2)$ with $k, u, v$ having the above. properties. The map $g$ is not surjective in the natural numbers, since the smallest values for $(k, u, v)$ are $(1, 1, 2) \mapsto 5$. If $u=1$, then $c=k(1 + v^2)$ where $v$ is even. For any $k$ of the form $(1 + 4p^2)$, we have that 
$$g(k, u, v) = g(1 + 4p^2, 1, v) = (1+(2p)^2)(1+v^2) = g(1+v^2, 1, 2p)$$
and $g$ is not injective. For example, $g(17, 1, 2) = 17(1+2^2) = 5(1+4^2) = g(5, 1, 4)$.  \\
\indent For the other two sides we have the maps 
\begin{align*}
g_a\colon (k, u, v) &\mapsto  k(v^2 - u^2) \\
g_b\colon  (k,u,v)  &\mapsto  2kuv
\end{align*}
where $h_a = g_a \circ f$ and $h_b = g_b \circ f$. Neither map is surjective, for 1 is not in the image of $g_a$ or $g_b$ over the range of allowable values $(k,u,v)$. By a similar argument as above $g_a$ is not injective, since if $k = 4p^2 -1$ then $g_a(k,1,v) = g_a(v^2-1, 1, 2p)$. Nor is $g_b$ injective, for if $2u_1v_1$ divides $u_2v_2$ then there is a $k$ such that $g_b(k,u_1,v_1) = g_b(1, u_2, v_2)$. An example is $g_b(10,1,2) = g_b(1,4,5)$.  

\exercise{Suppose we allow triangles to have negative side lengths. What differences does it make to the classification of Pythagorean triples?}
\solution
If $(a, b, c)$ is a Pythagorean triple with integer, possibly negative, values then so too is $(|a|, |b|, |c|)$ a Pythagorean triple, where each coordinate is the absolute value. Therefore it $(k, u, v)$ is such that $u, v$ are co-prime, $u+v$ odd and $u<v$, and the map $f\colon (k, u ,v) \mapsto (a,b,c)$ classifies positive integer Pythagorean triples, then
$$g: (a, b, c) \mapsto \{( (-1)^ia, (-1)^jb, (-1)^kc) \mid i, j, k \in \{0,1\}\}$$ 
is the map such that $g\circ f$ classifies Pythagorean triples with possibly negative side lengths. 

\exercise{How many Pythagorean triples have two sides differing by 1?}
\solution 
We have that 
$$c-b = k(u^2 + v^2) - 2kuv = k(u^2 + v^2 -2kuv) = k(v-u)^2$$ 
which is 1 whenever $k=1$ and $v = u+1$. In such cases $u$ and $v$ are co-prime, $u+v = 2u+1$ and $u<v$ so $(1,  u, u+1)$ generates a pythagorean triple with two sides differing by 1 for any integer $u >0$. There are infinitely many such triples. 

\newpage
\section{Specific-generality}

\exercise{Suppose we can prove that every map in which all countries have at least $k$ sides can be coloured with $k$ colours. Show that every map can then be coloured with $k$ colours.}
\solution
The proof is by induction. A map with less than $k$ countries can be colored with $k$ colors by picking a different color for each country. Suppose a map with $n$ countries can be colored with $k$ colors. Any map with $n+1$ countries such that each country has $k$ or more sides can be colored by assumption. Otherwise, if a map has $n+1$ countries and there is a country with $p$ sides, $p<k$, merging the country with one of its neighbors produces a map with $n+1$ countries which can be colored by the inductive hypothesis. When the merged country is reinserted into the colored map, the $p$ adjacent countries can have at most $p$ different colors, and therefore there are at least $k-p$ choices of colorings for the reinserted country that produce a valid $k$-colored map. We have shown that, under the general assumption that maps in which countries have at least $k$ sides can be colored, if any map with $n$ countries can be $k$-colored, then any map with $n+1$ countries can be $k$-colored. The result follows by induction.

\exercise{Show that if there are no solutions to the Fermat equation with $x, y, z$ pairwise co-prime then there are no solutions in general.}
\solution
Suppose 
$$x^n + y^n = z^n$$
is a solution to the Fermat equation, and there is some $p$ that divides two of $x, y, z$. Assume $p$ divides $x$ snd $p$ divides $y$. Then $x=pk$, $y=kl$ and
$$p^n(k^n + l^n) = z^n$$
so that $p$ divides $z$, and $z = pm$. Dividing by $p^n$,
$$k^n + l^n = m^n$$
which is a solution to the Fermat equation where $p$ is no longer a common divisor. This proves the contrapositive of the desired statement—if there is a solution to the Fermat equation in general, there is a solution with co-prime $x, y, z$.

\newpage
\section{Diagonal Tricks and Cardinality}
\exercise{Suppose $X$ and $Y$ are the natural numbers. We define $f$ to be multiplication by two from $X$ to $Y$. We define $g$ to be multiplication by three. What is the bijection $h$ constructed by the Schröder–Bernstein theorem? What if both maps are multiplication by 2?}
\solution When $f(x) = 2x$ and $g(x) =3x$ the Schröder–Bernstein theorem gives the bijection
$$h(x) = 
\begin{cases}
2x &\text{ if } 3\nmid x \text { or } 2\mid x \\
\frac{x}{3} &\text{ if } 3\mid x \text{ and } 2\nmid x
\end{cases}$$
and with $f(x) = 2x$ and $g(x) = 2x$ the bijection is 
$$h(x) = 
\begin{cases} 
2x &\text{ if } 2\nmid x \\
\frac{x}{2} &\text{ if } 2\mid x
\end{cases}.$$

%The Schröder–Bernstein theorem constructs a bijection $X\to Y$ from two injective maps $f\colon X\to Y$ and $g\colon X\to Y$ by partitioning $X$ into four sets under the equivalence relation $x_1 \sim x_2$ if $(g\circ f)^l (x_1) = x_2$ or $(g\circ f) (x_2) = x_1$ for some $l$. The first partition consists of those $x$ for which there exists $k>0$ such that $(g\circ f)^k (x) =x$. These points are considered to be an in infinite loop with respect to $(g\circ f)$. In terms of the functions on the natural numbers $f(x) = 2x$ and $g(x) = 3x$, we have that $(g\circ f) (x) = 6x > x$ for all $x \neq 0$. Then the first partition is just the set $\{0\}$. \\
%\indent The second partition consists of $x$ which are doubly infinite under the equivalence relation—that is, $x$ such that given $x_1 \sim x$ there exists $x_2\neq x_1$ for which $x_1 = (g\circ f) (x_2)$.  There are no sequences $[x]$ in the natural numbers such that there exists $x_1\in [x]$, $x_1 = 6x_2$ for every $x_2 \in [x]$, $x_1 \neq x_2$. This is because $\mathbb{N}$ is well ordered, and if $x_1 \neq x_2$ then $x_1 = 6x_2$ implies $x_1 < x_2$. Any subset of the natural numbers has a smallest element $x$ for which there exists no $x_2$ such that $x = 6 x_2$. Therefore the second partition is empty. \\
%\indent The third and fourth partitions consist of infinite sequences in $X$ generated by a point $x_0$. These are distinguished based on whether or not $x_0$ is in the image of $g$. Those $x_0$ that are not in the image of $g$ and for which $(g\circ f)^l (x_0)$ generates an infinite sequence are the natural numbers not divisible by 3. For if 3 does not divide $x$ then there is no $y$ such that $g(y) = 3y = x$. \\
%\indent The remaining natural numbers are in the fourth partition. These are generated by $x_0$ such that 3 divides $x_0$ but $6$ does not. Then there is no $x$ such that $(g \circ f ) x = x_0$ but there is a $y$ such that $g(y) = x_0$. \\
%\indent The bijection $h$ is constructed by defining $h(x) = f(x)$ if $x$ is in one of the first three partition, and $h(x) = g^{-1}(x)$ for $x$ in the fourth partition. This is the function 
%$$h(x) = \begin{cases} 2x &\text{ if }  x\in \{6^k x: 3  \nmid x\}\\\frac{x}{3} &\text{ if } x\in \{6^k x : 3 \mid x, 2\nmid x\}\end{cases}$$
%\indent Suppose $y\in Y$. If $g(y) = f(f\circ g)^m (x)$ for $x$ such that $g^{-1}(x)$ is not in the image of $f$, then $g(y)$ is in the fourth partition and 
%$$h(g(y)) = g^{-1}(g(y)) = y.$$
%otherwise $g(y)$ is in one of the other three cases and
%$$g(y) = (g\circ f) (x)$$
%for some $x$, and 
%$$y = g^{-1}(g(y)) = g^{-1}((g\circ f)(x)) =  f(x).$$
%We have that $h(x) = f(x)$ and $h$ is surjective. \\
%\indent Suppose that $h(x_1) = h(x_2)$. Since $f$ and $g^{-1}$ are injective, $f(x_1)\neq f(x_2)$ and $g^{-1}(x_1) \neq g^{-1}(x_2)$. So then $f(x_1) = g^{-1}(x_2)$ since $h$ is composed of $f$ and $g^{-1}$. But in that case, applying $g$ to both sides yields
%$$(g\circ f)(x_1) = x_2$$ 
%which shows they are in the same case, and are either both sent to $f$ or both sent to $g^{-1}$ by $h$. Then $h(x_1) \neq h(x_1)$. So $h$ is surjective and injective, and therefore $h$ is a bijection. 

\exercise{Show that every real number can be represented by a decimal that goes on forever.}
\solution
Let $x$ be a real number and $m$ be the least integer such that 
$$x < 10^m.$$
Let the sequence $a_n$ be defined such that 
$$x- \sum_{i=1}^{n} a_i10^{m-i} < 10^{m-n}.$$
Then for any $\epsilon$, there is a $n$ such that $\epsilon > 10^{m-n}$ and $x - \sum_{i=1}^n a_i10^{m-i} < 10^{m-n} < \epsilon$. Therefore $x$ is equal to the infinite sum specified by the sequence $\{a_n\}$, which is the decimal representation of $x$. 

\exercise{A submarine starts at integer point $n$. It travels with integer speed $k.$ So after turn $m$, its location is $n+mk$. A ship which does not know $n$ and $k$ drops a depth charge once a turn on an integer. Show that it is possible to design an algorithm so that the submarine is always hit eventually. What if the speeds are rational? What if they are real?}
\solution
The set of possible integer starting point and speed ordered pairs $(n, k)$ is the set $\mathbb{Z} \times \mathbb{Z}$. Since $\mathbb{Z}\times \mathbb{Z}$ is the product of countable sets, it is in bijection with the natural numbers. Let $h\colon \mathbb{N}\to \mathbb{Z}\times \mathbb{Z}$ be a bijection. At turn $m$, the function $f_{n,k}(m) = n +mk$ gives the location of the submarine for starting condition $(n, k)$. Since $h$ is surjective there exists $r\in \mathbb{N}$ such that $h(r) = (n, k)$. By dropping a depth charge on turn $m$ at the location $f_{h(m)}(m)$ the submarine will be hit on turn $r$. \\
\indent If the speeds are rational, then the set of starting conditions is $\mathbb{Q} \times \mathbb{Q}$ is still countable, and there is a bijection $g\colon \mathbb{Q}\times \mathbb{Q} \to \mathbb{N}$. So it is possible to design an algorithm using $g$ that will hit always the submarine with starting conditions $(n, k)$ after a finite number of turns given by $g(n, k)$. \\
\indent However if the starting conditions are real, the set of starting conditions has larger cardinality than the natural numbers, any any map $k\colon \mathbb{N} \to \mathbb{R}\times\mathbb{R}$ is not surjective. Therefore there is some starting condition not in the image of $k$ and it is not possible to design an algorithm that is guaranteed to hit the submarine.

\exercise{Consider the set of real-valued functions on the reals. What is the cardinality of this set?}
\solution 
A real valued function on the reals $f\colon \mathbb{R} \to \mathbb{R}$ can be described by an $\mathbb{R}$-dimensional vector with values in $\mathbb{R}$, a point in the space 
$$\prod_{\alpha \in \mathbb{R}} \mathbb{R} = \mathbb{R} \times \mathbb{R} \times \cdots.$$
\indent Let $\mathbb{R}^\mathbb{R}$ denote this set, and $F(\mathbb{R})$ the set of real-valued functions on the reals. Then we have $\varphi \colon F(\mathbb{R}) \to \mathbb{R}^\mathbb{R}$ such that 
$$\varphi(f) = \prod_{\alpha \in \mathbb{R}} f(\alpha).$$ 
\indent If $x\neq y \in \mathbb{R}^\mathbb{R}$ then for some $\alpha \in \mathbb{R}$ we have $x_\alpha \neq y_\alpha$, and the functions $f$ and $g$ for which $\varphi(f) =x$ and $\phi (g) = y$ are such that 
$$x = f(\alpha) \neq g(\alpha) = y$$
and $\varphi$ is injective. Any ordered tuple $K = (k_\alpha)\in\mathbb{R}^\mathbb{R}$ indexed by $\mathbb{R}$ corresponds to a function $h$, $h(\alpha) = k_\alpha$ such that $\varphi(h) = K$, so $\phi$ is surjective. Therefore $\phi$ defines a bijection and the cardinality of the set of real-valued functions on the reals is $\mathbb{R}^\mathbb{R}$. 

\exercise{Let $X$ be an infinite set show that $X\cup\mathbb{N}$ has the same cardinality as $X$.}
\solution Suppose $X$ is countably infinite. Then there is an bijection between $X$ and $\mathbb{N}$, and the elements of $X$ can be written $x_j$. Let the function $f$ be such that $f(x_j) = x_{2j}$ for $x\in X$ and $f(j) = x_{2j+1}$ for $j\in \mathbb{N}$. Then $f$ is an bijection from $X\cup \mathbb{N}$ to $X$ and the two sets have the same cardinality.  \\
\indent Suppose $X$ is uncountably infinite. Then there is no injection from $X$ to $\mathbb{N}$, but there is an injection $f\colon \mathbb{N}\to X$. Let $E\subset X$ be the image of $f$ and let $F$ be its complement in $X$ such that $X = E\cup F$. Since $E$ is countable we have just shown there exists a bijection $g\colon E\to E\cup \mathbb{N}$. Then the extension of $g$, $\bar g\colon E\cup F\to E\cup\mathbb{N}\cup F$ such that $\bar g$ is the identity on $F$, is a bijection between $X$ and $X\cup \mathbb{N}$ and the two sets have the same cardinality. 

\newpage
\section{Connectedness and the Jordan Curve Theorem}
\exercise{Show that a subset of $\mathbb{R}^n$ has $k$ components if and only if there exists an integer-valued continuous function that takes $k$ different values.} %2/17 
\solution
An integer-valued continuous function is constant on any straight line in $E\subset \mathbb{R}^n$. If $p$ and $q$ are in the same connected component of $E$, then there exists a path consisting of a union of straight line segments from $p$ to $q$. So an integer-valued continuous function $f\colon E\to \mathbb{Z}$ must have the property $f(p) = f(q)$. If $E$ has $k$ components then there are $k$ connected regions $E_k$ for which $f(p) = f(q)$ when $p, q \in E_k$. Let $f$ take the value $k$ on each $E_k$. Then $f$ is integer-valued continuous and takes $k$ different values. \\
\indent If there exists an integer-valued continuous function $g$ that takes $k$ different values on $E$, then there are $p_i, p_j\in E, 1\leq i, j\leq k$ such that $g(p_i)\neq g(p_j)$. By definition there must be no straight line between any $p_i$ and $p_j$. If the image of $f$ through $E$ is the set of integers $\{n_k\}$, then the sets $f^{-1}(n_k)\subset E$ for each $n_k$ are the $k$ connected components of $E$. For if $p$ and $q$ are connected then $f(p) = f(q)$ and $p,q\in f^{-1}(n_k)$ for some $k$. 

\exercise{Show that if a point, $p$, can be joined to a connected set, $U$, by a polygonal path then $p$ and all points of $U$ are in the same component} %2/17
\solution Let $r\in U$, then $q$ and $r$ are connected. If there is a path $P_{p,q}$ and a path $P_{q,r}$ then the union of the paths $P_{p,q} \cup P{q, r}$ produces a path from $p$ to $r$. Therefore $p$ is connected to every point in $U$ and they are in the same connected component.

\exercise{Will a finite intersection of connected sets be connected?} 
\solution Let $E_k$ be a finite collection of connected sets. If $\bigcap_k E_k$ is empty or consists of one point then it is trivially connected, so suppose $p, q \in \bigcap_k E_k$. Then $p$ and $q$ belong to $E_k$ for every $k$, and $p$ and $q$ are connected. 

\exercise{Suppose we have a collection of connected sets $U_j$ such that for all $j>1,$ there is some $k<j$, such that $$U_j\cap U_k \neq \emptyset,$$ does this guarantee that $\bigcup_j U_j$ is connected?} %2/17 
\solution Suppose there are 2 sets $U_j$. Then for $U_2$,  $U_2\cap U_1 \neq \emptyset$. Therefore $U_1$ and $U_2$ are connected. If there are $n+1$ sets $U_j$ and this property holds for up to $n$ sets, then $\bigcup_{j<n+1} U_j$ is connected, and $U_n \cap U_k \neq \emptyset$ for some $k<n$, so $U_n$ and $U_k$ are connected. Therefore $\bigcup_j U_j$ is connected. 

\exercise{Suppose we take the real line and subtract $k$ distinct points, how many components will there be? Prove it.} %2/17
\solution Let $R$ be the set $\mathbb{R} - \{p_k\}$. $R$ is divided into $k-1$ convex, connected intervals $(p_i, p_{i+1})$ for $1\leq i \leq k$. For if $x<y$ are in an interval $(p_i,p_{i+1})$, then the interval $(x,y)$ is in $(p_i,p_{i+1})$ and $x$ and $y$ are connected. Now $R - \{(p_i, p_{i+1})\}$ consists of the points less than $p_1$ and greater than $p_k$. If $x, y< p_1$ then there are no points removed in the interval $(x,y)$ and so $x$ and $y$ are connected. Likewise for $x, y > p_k$. Therefore $(-\infty, p_1)$ and $(p_k, \infty)$ are connected components and $R$ consists of $k+1$ components. 

\newpage
\section{The Euler Characteristic and the Classification of Regular Polyhedra}
\exercise{Suppose we make two $W$ shapes out of cubes and join the three top bits together, what happens?} %2/17, 2/18 
\solution The number of faces is decreased by $3\times 2 = 6$, the number of edges is decreased by $3\times 4 = 12$, and the number of vertices is decreased by $3\times 4 = 12$. The Euler characteristic becomes
$$\chi = \chi_1 + \chi_2 + ( (-12) -(-12) - 6) = 2 + 2 -6 = -2.$$ 

\exercise{If we attempt to apply our proof that the Euler characteristic is 2 to a ring shape, where does it fail?} %2/18
\solution A ring shape is not convex, so given a point $p$ inside the ring, the projection $\theta_p(q)$ from a point $q$ on the surface of the ring to a point on the surface of a sphere in which the ring is inscribed is not injective. The shape cannot be represented as a network of triangles in the plane.

\exercise{Suppose we take a square in the plane and cut it into small triangles, what is the Euler characteristic? Suppose we stick opposite sides together, what is the new Euler characteristic? What if we twist one pair of sides before gluing?}. %2/18
\solution A square has Euler characteristic
$$\chi = V - E + F = 4 - 4 +1 = 1.$$
\indent There are three ways to cut a square into smaller pieces with straight lines. 
\begin{enumerate}
\item Draw an edge between two non-adjacent vertices
\item Draw an edge between a vertex and an edge
\item Draw an edge between two edges 
\end{enumerate}
\quad \\
\indent In the first case, the number of edges increases by 1 and the number of faces increases by 1. In the second case, there are two added edges—one that was drawn and one from the edge that was bisected—one added vertex, and an additional face. In the third case, three edges, two vertices and one face are added. After applying these cuts to the square, there may still be non-adjacent vertices that share a face. If there are, they may be connected by drawing an edge until the shape consists only of triangles, adding one face and one edge each time. After any of these operations the Euler characteristic remains 1.\\
\indent For a square, sticking opposite sides together removes the 4 vertices and 4 edges on the boundary, and the Euler characteristic is 
$$\chi = 0 + 0 +1 = 1$$
which is unchanged. Then if the square is cut into triangles, the Euler characteristic depends on the triangles inscribed within the square, that is triangles that do not share an edge with the boundary. Any network of connected triangles has Euler characteristic 1, and gluing the boundary of the square together adds a face that is the complement of the are inside the network. Therefore the new shape has Euler characteristic 2. \\
\indent Deforming one pair of sides does not change the number of vertices, edges or faces, so the Euler characteristic after gluing is still 2. 

\exercise{Suppose we take a regular polyhedron and form a new one by making each vertex the center of one of the old faces and joining vertices from neighboring faces with edges. What happens for each of the 5 regular polyhedra?} %2/18
\solution The five regular polyhedra are 
\begin{enumerate}
\item Tetrahedron
\item Hexahedron (cube)
\item Octahedron
\item Dodecahedron 
\item Icosahedron
\end{enumerate}
\quad \\
\indent In the case of the tetrahedron, joining the centers of neighboring faces with edges creates a smaller tetrahedron with inverse orientation. \\
\indent Applying this process to the cube creates an octahedron. \\
\indent Likewise, applying this process to the octahedron creates a cube \\
\indent Joining the centers of a dodecahedron creates a icosahedron and joining the centers of an icosahedron creates a dodecahedron.

\newpage
\section{Discharging}
\exercise{What happens if we attempt to apply Thurston's discharging proof to a torus? (i.e. the surface of a doughnut.)} %2/20
\solution A torus is not convex, so it cannot be projected onto the surface of a sphere. 

\exercise{Is it possible to have a polyhedron whose faces are all pentagons or hexagons? How many pentagons must such a polyhedron have? Suppose we merge two adjacent faces what happens?} %2/20
\solution The formula is 
$$\sum_k (6-k) N_k = 12$$
where $k$ is the number of sides and $N_k$ is the number of faces with $k$ sides. So if the only $k$ is 5, we have 
$$(6-5)N_5 = 12$$
which shows that there must be 12 faces. This describes the character of a dodecahedron. If all of the faces have six sides, then 
$$(6-6)N_6 = 12$$ 
has no solution. If there are faces which are pentagons as well, then
$$(6-5)N_5 + (6-6)N_6 = 12$$
which means there may be any number of hexagons added, but there must be 12 faces which are pentagons. If two adjacent sides are merged, and they are both hexagons, the Euler characteristic doesn't change since the coefficient of $N_6$ is zero. Merging a pentagon with any other side however will change the Euler characteristic, and the resulting shape will not be a polyhedron. 

\newpage
\section{The Matching Problem}
 \exercise{The room-mate problem is to pair new students of the same sex into rooms. A pairing is stable if no two students prefer each other to their room-mate. Does the algorithm presented here apply?} %2/20
 \solution The matching algorithm is not applicable to this problem because it requires the population to be split into two sets such that a valid matching contains one member of each set. When all of the room-mates are the same sex the preference ranking for each student is over the whole set, and there is no way to non-arbitrarily split the students into two groups such that the matching algorithm will generate a stable pairing. 
 
 \exercise{We can rate a matching of all students by summing all the ranks of each member of a pair for his/her partner. Show that there is a stable pairing which minimizes this rating amongst stable pairings. Will it be unique? Design an algorithm to find it.} %2/20
 \solution Let $X$ and $Y$ be finite sets in a one two one correspondence and $S$ be the set of stable pairings. If $f\colon S\to \mathbb{N}$ is the sum of the ranks of each member of a pair for his or her partner then the image $f(S)$ is a subset of the natural numbers and has a minimum element $n$. Then $f^{-1}(n)$ is the non-empty set of stable pairings that minimize $f$. \\
 \indent However, the stable pairing that minimizes $f$ is not unique. In fact, every stable pairing has the same sum of rankings. for if $S$ is a pairing and there is are two pairs that if switched would create a lower sum of rankings, then the switched pair prefer each other to their assigned partners and the pairing $S$ is not stable.  
 
 \newpage
 \section{Games}
 \exercise{Suppose Nim is played with 2 stacks and each player can only take from one of them each turn. Analyze this game.} %2/20
 \solution If Nim is played with 2 stacks, and the winner is still the player who takes the last match, then the winning state is to have 1-3 matches on your turn. This is the case if your opponent has to choose from one stack 4 matches. Like regular Nim, if your opponent is left with one stack of $4k$ matches, you win. Assume stack 1 has $4k$ matches. If stack 2 has 1-3 matches on your turn, you win. Therefore if stack 2 has $4$ matches on your opponent's turn, you win. \\
 \indent The winning strategy for Nim with 2 stacks depends on the starting condition. If both stacks start with $4k$ matches, the second player wins by playing optimally: removing matches from whichever stack has $4k+ \{1, 2 , 3\}$ matches on your turn. If only one stack has $4k$ matches, the first player wins by removing matches from the other stack so that both stacks have $4k$ matches on the second player's turn. If neither stack has $4k$ matches, the problem reduces to the case where each stack has less than $4$ matches. \\
 \indent The first player wins if there is only one stack to take from, so the optimal strategy is to take from whichever stack is greater to leave the same number of matches on each stack. Therefore in the case when each stack 1 starts with $4k+n_1$ matches and stack 2 starts with $4k+n_2$, where $n_1, n_2 \in \{1,2,3\}$, the first player wins when $n_1\neq n_2$, and the second player wins otherwise. 
 \exercise{Suppose we play a version of Nim on a clock face. We start with the hour hand at 12 and the first player to get it to 6 wins. Each turn a player can move the hour hand 1–3 integer hours. What happens? What if we add in the rule that the clock cannot show the same time twice? What if we delete any time that has been visited by moving the remaining times closer together? What if we allow a 24-h clock?} %2/20
 \solution Assuming the clock can only move forward, the first player wins in all cases by moving the clock two hours, so assume the clock can be moved both clockwise and in the counter-clockwise directions. Under the first set of conditions, the game ends in a stalemate as each player can undo their opponent's move.\\
 \indent In the case that the clock cannot revisit the same time twice, notice that the second player wins when there are four previously visited times in the set $\{10, 11, 12, 1, 2\}$ and the clock hand is on the remaining time. Then the second player wins when there are two visited times that set and when there are zero. Since 10 and 2 are not reachable from either position, there are states in which the second player must make a specific move, but through optimal play the second player wins. When visited times are deleted, the restriction that 10 and 2 are mutually unreachable disappears and the second player still wins. \\
 \indent Allowing a 24 hour clock changes the set to $\{10, 11, 12, \dots, 24, 1, 2\}$. If 16 of the 17 times in this set have been visited and the clock hand is on the remaining time, the second to play wins. Since the set still has an odd number of elements, and after each pair of turns the number of unvisited elements is invariant, the second player wins when 0 times times have been visited and the clock is set to 12, i.e. at the starting conditions. 
 
 \exercise{Suppose the ability to ``pass'' in chess is added. If both players ``pass'' then the game is drawn. Can we then prove that there is not an optimal strategy making the second player always win?} %2/20
 \solution If there is an optimal strategy such that the second to play always wins, then the optimal strategy of the first player is to pass so that the second player is first to play. The optimal strategy for both players is to pass, so the game is drawn and by contradiction there is no optimal strategy making the second player always win.
 \exercise{There are $n$ lions in a cage. A piece of meat is thrown in. What happens under the following conditions?} %2/20
 \begin{itemize}
 \item The lions are hungry,
 \item A lion that eats the meat falls asleep.
 \item An asleep lion is meat to the other lions.
 \item Each lion and the meat is closest to exactly one other lion.
 \item The lions are ultra-intelligent.
 \item The lions prefer to stay alive.
 \end{itemize}
 \solution The winning state for any lion is to be the last lion in the cage with another lion who is asleep, at which point the winning lion both eats and stays alive. While there remains another lion that is not asleep, the optimal strategy is to pass, Since all the lions follow the optimal strategy, nothing happens. 
 
 \newpage
 \section{Analytical Patterns} 
 \exercise{Prove or disprove that the following sequences converge. If they converge, identify the limit.}%2/20 %2/25
 \begin{itemize}
 \item $x_k = k$.
 \item $x_k = 1-\frac{1}{k}$.
 \item $x_k = \frac{1+k}{2+k^2}.$
 \item $x_k = \sqrt{k+1} - \sqrt{k}$.
 \end{itemize}
 
\solution
\begin{enumerate}
\item Suppose $x_k\to x$. Then $x_k$ must get closer to $x$ as $k$ increases. Let $n$ be the closest integer to $x+1$, then for $k>n$ $x_k>n-k$ and the sequence doesn't satisfy the definition of convergence. 
 \item For this sequence $x_k$ converges to 1. We can write $x_k$ as the sum of the convergent sequences $y_k= 1, y_k\to 1$ and $z_k = \frac{1}{k} \to 0$. So $x_k = y_k + z_k \to y + z = 1$.
\item Dividing by $k$, the sequence is $x_k = \frac{1/k + 1}{2/k + k}$. The numerator converges to 1 and the denominator goes to infinity, so $x_K\to 0$. 
\item We can show that $x_k$ is convergent because it is a monotone bounded sequence. For the square root function, $\sqrt{x} > \sqrt{y}$ for all $x>y$, so $\sqrt{x+1}-\sqrt{x}$ is decreasing and bounded below by 0. Therefore it converges, and 0 is the greatest lower bound so it converges to 0. %not completed
 \end{enumerate}
\exercise{Suppose $f\colon \mathbb{R}^2 \to \mathbb{R}$, is continuous on the set $$E = \{1 \leq \vert (x,y) \vert \leq 2 \},$$ show that $f$ has a maximum and minimum on $E$. Find an example where these occur on the boundary.} %2/25
\solution The region $E$ represents the difference of a circle of radius $\sqrt 2$ and of radius 1 in the plane. This can be split up into the regions where $\{1\leq |(x,y) \leq 1.5\}$ and $1.5 \leq |(x,y)| \leq 2|$. Now if one region has an $(x',y')$ such that $f(x', y') < f(x,y)$ for $(x,y)$ in the  other region, then let $(x_0,y_0)$ be that $(x', y')$ and split the region in two and repeat the process. This generates a sequence $(x_i,y_i)$, and continuity of $f$ gives that $f(x_i,y_i) \to f(x,y)$ which is the maximum. The same process gives the minimum by taking the infimum over $f$ of each of the two regions at each step. \\
The maximum and minimum occur on the boundary for any radially monotonic function $f$. So for example $f(x,y) = x^2 + y^2$, as the radius increases, $\sqrt{x^2 + y^2}$ increases and so $f$ increases. Then the minima of $f$ on $E$ occur for any $(x,y)$ such that $|(x,y)| = 1$ and the maxima at $|(x,y)| = 2$.  

\exercise{A function is said to have the intermediate value property on $\mathbb{R}$ if $f\colon \mathbb{R}\to\mathbb{R}$, and if $a< b, f(a) < x < f(b)$ implies that there exists $c\in (a,b)$ with $f(c) = x$. Let a function $g$ have value 0 at 0 and value $$g(x) = \sin{\frac{1}{x}}$$ for $x\neq 0$.  Does $g$ have the intermediate value property? Is $g$ continuous?}
\solution The function $g$ does have the intermediate value property but it is not continuous at zero, since the sequence $\frac{1}{n}\to 0$ but $g(1/n)$ doesn't converge. 

\exercise{Does each of the following series converge? Prove or disprove.}
\begin{itemize}
\item $x_k = \frac{1}{1+k}$. 
\item $x_k = \frac{1}{1+k^2}$.
\item $x_k = \frac{1} {1+k^3}$.
\end{itemize}
\solution 
\begin{enumerate}
\item The harmonic series $a_n = 1/n$ doesn't converge, and $x_k = a_{n+1}$ so the series have the same terms. Therefore the $x_k$ doesn't converge.  
\item Each term of this series is less than the terms of the convergent series $\frac{1}{n^2}$, so it converges.
\item Each term of this series is less than the terms of the convergent series $\frac{1}{n^3}$, so it converges.
\end{enumerate}

\newpage
\section{Counterexamples} %completed 2/25
\exercise{Find matrices $A$ and $B$ which are distinct and $AB = BA \neq 0$.}
\solution $A = \begin{bmatrix} 1 & 0 \\ 0 & 2\end{bmatrix}.$ $B = \begin{bmatrix} 2 & 0 \\ 0 & 1 \end{bmatrix}.$ $AB = BA = \begin{bmatrix} 2 & 0 \\ 0 & 2\end{bmatrix}. $
\exercise{Find a matrix will all entries non-zero which is not invertible.}
\solution $\begin{bmatrix} 1 & 1 \\ 1 & 1 \end{bmatrix}$
\exercise{Do there exist square matrices, $A$, with $A^2 = 0$ and all entries nonzero?}
\solution Such a matrix gives the equations $a^2 = -bc$, $ab = -bd$, $ac = -cd$, and $d^2 = -bc$. If all entries are non-zero. then $a^2 = d^2$ and $a = -d$. Such a matrix: $A = \begin{bmatrix} 1 & 1 \\ -1 & -1\end{bmatrix}$ and $A^2 = 0$.
\exercise{If $A$ is a square matrix and there exists $B$ such that $AB = I$, must there exist $C$ such that $$CA = I,$$ and if $C$ does exist, must we have $C = B$?}
\solution Suppose $C=BA $, then $CB=BAB = BI = B$ so that $C = I$ and so $BA=I$. If there exists another $C$ such that $CA = I$ then $CAB = B$ and therefore $C=B$.
\exercise{Which of the following functions are Lipschitz continuous?}
\begin{itemize}
\item $f(x) = x^{2/3}$.
\item $f(x) = x.$
\end{itemize}
\solution 
\begin{enumerate}
\item Let $t=0$ in $|s^{2/3} -t^{2/3}|$, then if $|s|$ is arbitrarily small $|s^{2/3}|$ is arbitrarily bigger than $|s|$. So there can't exist a constant $C$ where $|f(s)-f(t)| \leq C|s-t|$ and $f$ is not Lipschitz continuous.
\item Since $|s-t| < C|s-t|$ for any $C>1$, $f$ is Lipschitz continuous. 
\end{enumerate}
\exercise{Suppose $f:(0,1)\to \mathbb{R}$ is continuous. Must there be a global maximum in (0,1)? i.e. does there always exist $x$ such that $f(y) \leq f(x)$ for all $y\in (0,1)$?}
\solution No, consider the function $f(x) = \frac{1}{x}$. 
\end{document}
